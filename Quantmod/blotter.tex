
blotter {blotter}	R Documentation
Tools for Transaction-Oriented Trading Systems Development

Description

Transaction-oriented infrastructure for constructing transactions, portfolios and accounts for trading systems and simulation. Provides support for multi-asset class and multi-currency portfolios for backtesting and other financial research. Still in heavy development.

Details

The blotter package provides infrastructure for developing trading systems and managing portfolios in R. Although the name might suggest a smaller scope, blotter provides functions for tracking trades and positions in portfolios, calculating profit-and-loss by position and portfolio, and tracking performance in a capital account.

Blotter works with a companion package, FinancialInstrument , that defines meta-data for tradeable contracts (referred to as "instruments", e.g., stocks, futures, options, etc.), so that blotter can support multi-asset portfolios, derivatives and multiple currencies. As used here, instruments are objects that define contract specifications for a tradable contract, such as IBM common stock. These objects include descriptive information and attributes that help identify and value the contract.

Blotter's scope is focused on the heirarchy of how transactions accumulate into positions, then into portfolios and an account. 'Transactions' are typically trades in an instrument - an amount bought or sold at a price and time. But other transaction types include splits, dividends, expirations, assignments, etc. (some of which are implemented currently, others are not).

Those transactions are aggregated into 'positions'. Positions are held in a 'portfolio' that contains positions in several instruments. Positions are valued regularly (usually daily) using the price history associated with each instrument. That results in a position profit-and-loss (or PnL) statement that can be aggregated across the portfolio.

Portfolios are associated with an 'account'. An account is modeled as a cash account where investments, withdrawals, management fees, and other capital account changes are made.

The package also contains functions to help users evaluate portfolios, including charts and graphs to help with visualization. A small selection of post-trade metrics has been added recently, and is likely to change and expand as we hear more about what people want to calculate.

Blotter's functions build and manipulate objects that are stored in an environment named ".blotter" rather than the global environment, .GlobalEnv. Objects may be listed using ls(envir=.blotter). See environment for more detail.

We do that for two reasons. First, keeping them out of the .GlobalEnv means less clutter in the user's workspace and less chance of clobbering something locally. Second, we don't recommend acting on the account and portfolio objects directly. Manipulating the objects directly will almost certainly create inconsistencies and problems with the resulting calculations. Instead, we recommend copying them into the local workspace using getPortfolio or getAccount functions, or simply using blotter functions.

Blotter relies heavily on xts for the heavy lifting of date and time subsetting. Many thanks to Jeff Ryan and Josh Ulrich for their diligent help ferretting out all those corner cases. As a result, any Dates parameter in the package leverages xts date scoping. For example, getTxn(Portfolio="p", Symbol="XYZ", Dates="2007-01") will retrieve all transactions for "XYZ" from January, 2007. Similarly, a range of dates may be specified as "2007-01::2008-04-15". Take a look at the xts documentation for more detail. Otherwise, the Date or Dates parameter accepts an ISO 8601 format such as '2008-09-15' or '2010-01-05 09:54:23.12345'. In almost all cases, if the Dates parameter is unspecified, the function will return all results.

The sole example in the documentation of blotter follows, providing a simple but more complete overview of how blotter can be used. More extensive examples can be seen in the demos. The longtrend demo shows a simple trend-following example, and turtles shows a more complicated trend-following example. The amzn_test demo shows blotter's use with tick data.

Author(s)

Peter Carl, Brian Peterson

Maintainer: Brian Peterson brian@braverock.com

See Also

quantmod, xts, PerformanceAnalytics

Examples

## Not run: 
# Construct a portfolio object and add some transactions
## These two lines are here to deal with frame issues in R CMD check
## and ARE NOT NECESSARY to run by hand in your own environment.
if(!exists(".instrument")) .instrument <<- new.env()
if(!exists(".blotter")) .blotter <<- new.env()

# Use the FinancialInstrument package to manage information about tradable
# instruments
require(FinancialInstrument)
# Define a currency and a couple stocks
currency("USD")
symbols = c("IBM","F")
for(symbol in symbols){ # establish tradable instruments
    stock(symbol, currency="USD", multiplier=1)
}

# Download price data
require(quantmod)
getSymbols(symbols, from='2007-01-01', to='2007-01-31', src='yahoo',
index.class=c("POSIXt","POSIXct"))

# Initialize a portfolio object 'p'
print('Creating portfolio \"p\"...')
initPortf('p', symbols=symbols, currency="USD")

## Trades must be made in date order.
print('Adding trades to \"p\"...')
# Make a couple of trades in IBM
addTxn(Portfolio = "p", Symbol = "IBM", TxnDate = '2007-01-03', TxnQty = 50,
TxnPrice = 96.5, TxnFees = -0.05*50)
addTxn("p", "IBM", '2007-01-04', 50, 97.1, TxnFees = -0.05*50)

# ...a few in F...
addTxn("p", "F", '2007-01-03', -100, 7.60, TxnFees = pennyPerShare(-100))
addTxn("p", "F", '2007-01-04', 50, 7.70, TxnFees = pennyPerShare(50))
addTxn("p", "F", '2007-01-10', 50, 7.78, TxnFees = pennyPerShare(50))
# pennyPerShare is an example of how a cost function could be used in place of
# a flat numeric fee.

# ...and some in MMM
# We didn't include this in the list of symbols for the portfolio, so first we
# have to download prices and add a slot for MMM to the portfolio
getSymbols("MMM", from='2007-01-01', to='2007-01-31', src='yahoo',
index.class=c("POSIXt","POSIXct")) # Download price data
stock("MMM", currency="USD", multiplier=1) # Add the instrument

# Now we can add transactions:
# TODO: convert this to addTxns
addTxn("p", "MMM", '2007-01-05', -50, 77.9, TxnFees = -0.05*50)
addTxn("p", "MMM", '2007-01-08', 50, 77.6, TxnFees = -0.05*50)
addTxn("p", "MMM", '2007-01-09', 50, 77.6, TxnFees = -0.05*50)

print('Updating portfolio \"p\"...')
updatePortf(Portfolio="p",Dates='2007-01')

print('Creating account \"a\" for portfolio \"p\"...')
initAcct(name="a", portfolios="p", initEq=10000, currency="USD")
print('Updating account \"a\"...')
updateAcct("a",'2007-01') # Check out the sweet date scoping. Thanks, xts.
updateEndEq("a",'2007-01')
PortfReturns(Account="a",Dates="2007", Portfolios="p")
# Examine the contents of the portfolio
## Here is the transaction record
getTxns(Portfolio="p", Symbol="MMM", Date="2007-01")
getTxns(Portfolio="p", Symbol="MMM", Date="2007-01-03::2007-01-05")
## Here are the resulting positions
getPos(Portfolio="p", Symbol="MMM", Date="2007-01")
getPos(Portfolio="p", Symbol="MMM", Date="2007-01-05")
getPosQty(Portfolio="p", Symbol="MMM", Date="2007-01")

# Alternatively, you can copy the objects into the local workspace
p = getPortfolio("p") # make a local copy of the portfolio object
a = getAccount("a") # make a local copy of the account object

chart.Posn(Portfolio="p", Symbol="MMM", Dates="2007-01")

# add tradeStats

## Not Run
## Here is the transaction record in the local object
p$symbols$MMM$txn

## Here is the position and any gains or losses associated in the local currency
## and the portfolio currency (which are the same in this example)
p$symbols$MMM$posPL
p$symbols$MMM$posPL.USD

## Here is the calculated portfolio summary denominated in the portfolio
## currency
p$summary

## End(Not run)